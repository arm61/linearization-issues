% Define document class
\documentclass[reprint,superscriptaddress,nobibnotes,amsmath,amssymb]{revtex4-2}
\usepackage{showyourwork}
\usepackage[version=4]{mhchem}
\usepackage{graphicx}
\usepackage{siunitx}

% Begin!
\begin{document}

% Title
\title{Is there still a place for linearisation in the chemistry curriculum?}

% Author list
\author{Andrew R. McCluskey}
  \email{andrew.mccluskey@ess.eu}
  \affiliation{European Spallation Source ERIC, Ole Maaløes vej 3, 2200 København N, DK}

% Abstract with filler text
\begin{abstract}
    Lorem ipsum dolor sit amet, consectetuer adipiscing elit.
    Ut purus elit, vestibulum ut, placerat ac, adipiscing vitae, felis.
    Curabitur dictum gravida mauris, consectetuer id, vulputate a, magna.
    Donec vehicula augue eu neque, morbi tristique senectus et netus et.
    Mauris ut leo, cras viverra metus rhoncus sem, nulla et lectus vestibulum.
    Phasellus eu tellus sit amet tortor gravida placerat.
    Integer sapien est, iaculis in, pretium quis, viverra ac, nunc.
    Praesent eget sem vel leo ultrices bibendum.
    Aenean faucibus, morbi dolor nulla, malesuada eu, pulvinar at, mollis ac.
    Curabitur auctor semper nulla donec varius orci eget risus.
    Duis nibh mi, congue eu, accumsan eleifend, sagittis quis, diam.
    Duis eget orci sit amet orci dignissim rutrum.
\end{abstract}

\maketitle 

The ``linearisation'' of non-linear functions is a popular tool in chemical education. 
This is where a non-linear function, that describes the dependency of a chemical observable on some variable, is turned into a linear one, through some mathematical transformation. 
This process allows linear regression to be used to quantify the relationship between the dependent and independent variables.

Linearisation is commonly seen in chemical kinetics courses, as it is useful in determining rate constants and Arrhenius relationships \cite{perrin_linear_2017,hites_calculating_2017}.
For example, if we consider a second-order integrated rate law,
%
\begin{equation}
    \frac{1}{[\mathrm{A}]_t} = \frac{1}{[\mathrm{A}]_0} + kt,
\end{equation}
%
where, $k$ is the rate constant, $[\mathrm{A}]_t$ is the concentration of reactant \ce{A} at time $t$, and $[\mathrm{A}]_0$ is the initial concentration of \ce{A}. 
Plotting $1/[\mathrm{A}]_t$ against $t$ linearises this function to give a straight line, the gradient and intercept of which are the rate constant and reciprocal of the initial concentration, respectively. 
Alternatively, the Arrhenius equation, 
%
\begin{equation}
    k = A\exp{\bigg(-\frac{E_\mathrm{a}}{RT}\bigg)},
\end{equation}
%
is linearised by taking the natural logarithm of both sides to obtain
%
\begin{equation}
    \ln{(k)} = -\frac{E_{\mathrm{a}}}{R}\frac{1}{T} + \ln{(A)},
\end{equation}
%
where, $E_{\mathrm{a}}$ is the activation energy, $R$ is the ideal gas constant, $T$ is the temperature, $A$ is the pre-exponential factor and, again, $k$ is the rate constant.
Plotting $\ln{(k)}$ against $1/T$ allows the activation energy to be found as the gradient multiplied by $-R$ and the pre-expoential factor is the exponential of the intercept.

In theory, the process of linearisation should have no impact on the resulting values of our ``fitted'' parameters (i.e. the rate constant or activation energy). 
However in practice, our measurements are not perfect and are only estimates of the true value of the dependent variable.
These estimates are random draws from some distribution of values for the dependent variable that depends on the measurement method. 
However, the use of linearisation makes inappropriate assumptions about these data, leading to biased estimates to the fitted parameters. 

% Measurement of concentration by UV-vis and why this will be normally distributed.
Let us consider, as a representative example, the measurement of the concentration of a chemical reactant, \ce{A}, as a function of time, where \ce{A} is participating in a second-order reaction, i.e. \ce{A + B -> C}. 
We can measure the concentration of \ce{A} over time by UV-visible spectroscopy, where the absorbance, $A$, of some wavelength of light is proportional to the concentration, following the Beer-Lambert law, such that the second-order integrated rate law becomes
%
\begin{equation}
    \frac{1}{A(t)} = \frac{1}{A(0)} + kt.
\end{equation}
%

If we were to repeat the measurement of the absorbance multiple times at a single $t_x$, we would develop a distribution of potential values of $A(t_x)$.
The shape of this distribution would depend on the sources of uncertainty in the measurement, which can be either systematic or random. 
We will imagine that we have no systematic uncertainty in our apparatus and therefore all uncertainty is random in nature. 
It is standard to assume these random uncertainties follow a normal (or Gaussian) distribution. 
Therefore, any one of our measurements of $A(t_x)$ is a random draw from this normal distribution.
Typically, we aim to quantify the standard deviation of this normal distribution with our measurement uncertainty, the error bars if we are to plot the data. 

% An example of a second order rate law fitting failing
The rate constant can be estimated from the measurement of $A$ at a range of times as the reaction proceeds.
Then, the common practice, is to plot the linear form of this data (Figure~\ref{fig:second_order}b) and fit a straight line. 
If we were to repeat the measurement of $A(\SI{600}{\second})$ $2^{15}$ times with a random uncertainty, that is constant for all measurements of $0.04$, the normal distribution shown in Fig.~\ref{fig:second_order}c would be observed. 
Taking the reciprocal of this normal distribution will produce a distribution that is no longer normal (Fig.~\ref{fig:second_order}d). 
%
\begin{figure}
  \includegraphics[width=\columnwidth]{figures/second_order.pdf}
  \caption{
    Plots.
    }
  \label{fig:second_order}
  \script{second_order.py}
\end{figure}
%

Linear regression, ideally weighted~\cite{perrin_linear_2017}, is commonly used to find the gradient and intercept of a linearised relationship. 
Linear regression makes the assumption that the observations are drawn from normal distributions. 
Therefore, using the draw from a reciprocal of a normal distribution will lead to biased estimates of the regression parameters.
Fig.~\ref{fig:fit_second} shows the result of performing non-linear and linear fits to $2^15$ datasets (one of which is shown in Fig.~\ref{fig:second_order}). 
When the non-linear fitting is performed, the distribution of estimated values of the rate constant is normal with a mean value approximately equal to the true value of $k$ ($E(k) - k = \variable{output/non_bias.txt}$).
Whereas, the use of a linear fit approach results in a non-normal distribution, where the difference between the mean and true rate constant, the bias, is \variable{output/lin_bias.txt}, \variable{output/bias_ratio.txt} times larger than for the non-linear approach. 
%
\begin{figure}
  \includegraphics[width=\columnwidth]{figures/fit_second.pdf}
  \caption{
    Plits.
    }
  \label{fig:fit_second}
  \script{fit_second.py}
\end{figure}
%

% An example of a Arrhenius plot fitting failing
This problem is much m

% Some concluding paragraphs

There is important historical and pedagogical content behind the use of linearisation. 
In the classroom or exam hall, it is feasible for a student equipped with graphing paper and a ruler to estimate the gradient of a straight line, while the estimation of a parameters of a non-linear function is less achievable. 
However, these shortcuts propagate into formal analysis, leading to estimates of parameters from linearisation regularly being found in scholarly work. 
I question if, with the availabilty of computers that can trivially perform non-linear fitting, the benefits of simplicity outweigh the problems of inaccuracy.


\bibliography{bib}

\end{document}
